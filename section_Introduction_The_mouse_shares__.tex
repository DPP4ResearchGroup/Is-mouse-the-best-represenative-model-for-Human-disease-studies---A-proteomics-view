\section{Introduction}

The mouse shares its majority protein encoding genes with human and is arguably one of the most popular research models for human disease. Wide spread use of mouse models in biomedical research can not be understated. \cite{MORSEIII_2007} One significant benefits of utilizing a rodent model is fast and economical breading in captivity \cite{Rosenthal_2007}, which allows fast tract experimental results. \cite{Vandamme_2014} Without the question, mouse (also other rodents) has many physiological features and pathological pathways that are very similar to human including vital nervous, cardiovascular system and immune system, which make 

Dispite the dominant application of mouse models in biomedical researches, recent studies have also highlighted the high degree divergence between human and mouse genome in regarding to transcription regulations. \cite{25409824} This posts an important question about whether default mouse models are the best representation to study all human diseases. In this article, we are using bioinformatic pipeline to examine the substrate degradome difference between human and mouse in the context of dipeptidyl dipeptidase-4 (DPP4).  


\cite{Justice_2011}