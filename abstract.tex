\ Regardless popular demand for mouse models in biomedical research, concerns regarding whether mouse model is appropriate for all human disease studies persist. Mouse and human shares many similar physiological and molecular features. Nevertheless, the distinct difference between the two spices at molecular level can not be ignored, particular high attrition rate in phase II new drug clinical trails highlight the lack of comprehensive understanding of molecular and physiological difference between human and mouse. In this study, we have used bioinformatics approach to depict the difference of substrate degradome of DPP4 between human and mouse to gain better understand if mouse model is a good representation of all DPP4 features. DPP4 is an important exo-protease and regulates many important biological molecules including regulating the immune system response, cellular homeostasis and DPP4 is a known therapeutic target for managing type-2 diabetes. We reveals that only two third of human DPP4 substrate degradome has mouse representation, yet only one third of human DPP4 substrates have cleavable mouse representation, which highlights the mouse model may not full reflect the DPP4 physiology in human and urges new studies to reveal DPP4 functions as a protease regulator in human context. 
