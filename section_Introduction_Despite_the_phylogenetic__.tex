\section{Introduction}

Despite the phylogenetic divergence between human and mouse occurred approximately 90 Ma ago \cite{Hedges_2006}, two species kept a close evolutionary affinity, which makes mouse a good representative models of human phenotypical traits. In fact, the mouse is arguably one of the most popular research models for human disease, with over 25 million mice have been breed for and shipped to research laboratories \cite{Rosenthal_2007}. For the past half century, mouse has been used to advance our knowledge on mammalian development \cite{Ueda_2006, Cheon_2011}, to test the new drugs \cite{Van_Dam_2011} and many more. The wide spread use of mouse models in biomedical research can not be understated. \cite{MORSEIII_2007} One significant benefits of utilizing a rodent model is fast and economical breading in captivity \cite{Rosenthal_2007}, which allows fast tract experimental results. \cite{Vandamme_2014} Without the doubts, mouse (also other rodents) has many physiological features and pathological pathways that are very similar to human including vital nervous, cardiovascular and immune system. \cite{MORSEIII_2007,Rosenthal_2007} In addition, large-scale standards including mouse phenotyping protocols and bulk immediate available referencing, mature and efficient genetic modification technology particular the emergence of CRISPR/Cas9, good ethical approval rates all contributed to mouse models' popularity in biomedical research \cite{26214591, Rosenthal_2007}. 
\\

As biomedical researches march into molecular era, we start to acquire significant knowledge in genetics and proteomics for both human and mouse. Two species no-surprisingly shares many molecular features. In particular, mouse and human share significant proteome identity, where the mouse has its majority (78.5\%) amino acid sequences aligned to human \cite{Lindblad_Toh_2001}. Furthermore, mouse and human also have strong homology in many important biomolecules including immune receptors. For instance, study on macrophage-inducible Ca\textsuperscript{2+} dependent lectin (also known as mincle) \cite{Rambaruth_2015}, which is associated with the recognition of glycans on pathogens \cite{Ishikawa_2009, 18490740, Yamasaki_2009} and is important for immune response against pathogen invasion, has shown the conservation in homology and function between human and mouse. Large scale microarray analysis also revealed a close gene expression pattern between human and mouse in general \cite{Liao_2005}. More organism wide gene expression studies have also emphasised the conservation, including tissue specific gene expression study \cite{Zheng_Bradley_2010} concluded the similar gene expression profiles between mouse and human in muscle, liver and nervous cells. With increasing use of gene co-expression maps, it is possible to determine the degree of conservation between species at epigenetic level \cite{Stuart_2003,Oldham_2006}. Study in Alzheimer's disease \cite{Miller_2010} has insured a high degree similarity in gene interactions between human and mouse brains and recent gene co-expression study \cite{Monaco_2015} has also reassured the strong conservation of gene expression in the bone and those associated with fundamental cellular functions like cell adhesion, cell cycles, DNA replication and DNA repaires. \cite{Monaco_2015} At proteomics level, recent study on mouse and human saliva proteome \cite{Karn_2013} has identified the conservation in core proteins that involves in major functions including early stage of digestion, and protecting and lubricating the cavity surface. The protein-protein interaction (PPI) network is a direct representation of biological mechanisms and according to recently published integrated interactions database \cite{Kotlyar_2015}, PPI networks in human from 29 tissues are closely represented in mouse in general, with over 85\% human PPI has mouse representation at ortholog level. However, it is also worth to mention that only limited amount of claims regarding to the mouse and human molecular similarity have been backed up with multiple studies except the conservation of gene expression in brains \cite{Liao_2005, Voolstra_2006, Miller_2010, Chan_2009}. 
\\

Despite high degree of conservation in many molecular features, human and mouse are also presented very differently in many molecular fronts. For instance, the gene expression in reproductive system like testis \cite{Chan_2009, Brawand_2011, Necsulea_2014} and high redundancy in gene functions that are associated with smell in mouse \cite{Gilad_2009, Gilad_2003, Young_2002} are among those high degree of divergence areas. Recent studies have addressed the high degree divergence between human and mouse genome in regarding to transcription regulations. \cite{25409824} In addition, transcriptome study on erythroid-specific long non-coding RNA has also revealed the high degree divergence between human and mouse \cite{An_2015}, which hindered the large difference in gene regulation between the two spices. At proteomics level, the presence of unique mouse saliva proteome compare to human are suspected to relate to but not limited to the behaviour difference between two spices including grooming and pelage maintenance \cite{Karn_2013}, this posted a threat of misleading information presented by mouse models in reflecting human oral physiology. Moreover, high attrition rate on new drug testing during Phase II clinical trials in recent years \cite{Arrowsmith_2011} and occasionally even reverse effects were observed. This development signifies the lack of comprehensive understanding of molecular and physiological difference between human and mouse \cite{de_Magalh_es_2014} and posts the doubts about whether default mouse models are the best option for studying all human diseases. In this article, we are using bioinformatic pipeline to examine the substrate degradome difference between human and mouse in the context of dipeptidyl dipeptidase-4 (DPP4), in light to answer the question whether mouse is a good animal model to study DPP4 physiology in human.
\\

DPP4 is exo-protease in nature and is able to cleave first two amino acid from substrate's N-terminus. However, DPP4 is a unique protease, in which it has rare ability to cleave evolutionary conserved penultimate proline. From our current knowledge, human and mouse's DPP4 has conserved catalytic specificity. Here poses a question, if many human DPP4 substrates do not have an exact representation in mouse, particularly if N-terminus is not conserved, a deemed DPP4 substrate in mouse model is not likely to be a human substrate. Despite the many years research in DPP4 research, there is limited \textit{in vivo} evidence has been depicted using mouse model. Maybe, just maybe, mouse is not a perfect model for studying DPP4's pathological role in human. \\

Diabetes mellitus is a major health risk affecting global populations with predicted patients exceed 590 million by 2035. \cite{IDF_Diabetes_Atlas_Group_2015} In order to address the urgent needs to improve the efficacy of medical intervention, many mouse models have been used to understand the disease mechanisms. However, the predictive therapeutic values based on mouse models have been less significant \cite{Renner_2016, Hay_2014}. Recent research in anaimal models 

\cite{Justice_2011}