\section{Introduction}

Despite the phylogenetic divergence between human and mouse occurred approximately 90 Ma ago \cite{Hedges_2006}, two species kept a close evolutionary affinity, which makes mouse a good representative models of human phenotypical traits. In fact, the mouse is arguably one of the most popular research models for human disease. For the past half century, mouse has been used to advance our knowledge on mammalian development \cite{Ueda_2006, Cheon_2011}, to test the new drugs \cite{Van_Dam_2011} and many more. The wide spread use of mouse models in biomedical research can not be understated. \cite{MORSEIII_2007} One significant benefits of utilizing a rodent model is fast and economical breading in captivity \cite{Rosenthal_2007}, which allows fast tract experimental results. \cite{Vandamme_2014} Without the doubts, mouse (also other rodents) has many physiological features and pathological pathways that are very similar to human including vital nervous, cardiovascular and immune system. \cite{MORSEIII_2007,Rosenthal_2007} \\

As biomedical researches march into molecular era, we start to acquire significant knowledge in genetics and proteomics for both human and mouse. Two species no-surprisingly shares many molecular features. In particular, the mouse shares its majority (78.5\%) protein amino acid identity with human \cite{Lindblad_Toh_2001}. Large scale microarray analysis also revealed a close gene expression pattern between human and mouse in general \cite{Liao_2005}. In addition, Tissue specific gene expression study \cite{Zheng_Bradley_2010} also highlights the conservation of gene expression profiles in muscle, liver and nervous cells. 

Nevertheless, human and mouse are also presented very differently in many molecular fronts. 

With increasing use of gene co-expression maps, it is possible to determine the degree of conservation between species at epigenetic level \cite{Stuart_2003,Oldham_2006}. Study in Alzheimer's disease \cite{Miller_2010} has insured a high degree similarity in gene interactions between human and mouse brains. 

Recent gene co-expression study \cite{Monaco_2015} has reassured the strong conservation in gene expression in the brain, bone and gene associated with fundamental cellular functions like cell adhesion, cell cycles, DNA replication and DNA repaires. \cite{Monaco_2015} Nevertheless, there are limited amount of work has been backed up with multiple studies except the conservation of gene expression in brains \cite{Liao_2005, Miller_2010, Chan_2009}, high degree of divergence in gene expression in reproductive system like testis 
\\

Dispite the dominant application of mouse models in biomedical researches, the attrition rate on new drug testing during Phase II clinical trials remains high in recent years \cite{Arrowsmith_2011} and sometimes even reverse effects were observed. This significance highlighted the lack of comprehensive understanding of molecular and physiological difference between human and mouse \cite{de_Magalh_es_2014}. Recent studies have addressed the high degree divergence between human and mouse genome in regarding to transcription regulations. \cite{25409824} In addition, transcriptome study on erythroid-specific long non-coding RNA has also revealed the high degree divergence between human and mouse. \cite{An_2015} This posts an important doubt about whether default mouse models are the best option for studying all human diseases. In this article, we are using bioinformatic pipeline to examine the substrate degradome difference between human and mouse in the context of dipeptidyl dipeptidase-4 (DPP4).  \\

DPP4 is exoprotease in nature and is able to cleave first two amino acid from substrate's N-terminus. However, DPP4 is a unique protease, which has rare ability to cleave penultimate proline. From our current knowledge, human and mouse's DPP4 has conserved catalytic specificity. Here poses a question, if many human DPP4 substrates do not have an exact representation in mouse, particularly if N-terminus is not conserved, a deemed DPP4 substrate in mouse model is not likely to be a human substrate. Despite the many years research in DPP4 research, there is limited \textit{in vivo} evidence has been depicted using mouse model. Maybe, just maybe, mouse is not a perfect model for studying DPP4's pathological role in human. 

\cite{Justice_2011}